\documentclass[]{article}
\usepackage{lmodern}
\usepackage{amssymb,amsmath}
\usepackage{ifxetex,ifluatex}
\usepackage{fixltx2e} % provides \textsubscript
\ifnum 0\ifxetex 1\fi\ifluatex 1\fi=0 % if pdftex
  \usepackage[T1]{fontenc}
  \usepackage[utf8]{inputenc}
\else % if luatex or xelatex
  \ifxetex
    \usepackage{mathspec}
  \else
    \usepackage{fontspec}
  \fi
  \defaultfontfeatures{Ligatures=TeX,Scale=MatchLowercase}
\fi
% use upquote if available, for straight quotes in verbatim environments
\IfFileExists{upquote.sty}{\usepackage{upquote}}{}
% use microtype if available
\IfFileExists{microtype.sty}{%
\usepackage{microtype}
\UseMicrotypeSet[protrusion]{basicmath} % disable protrusion for tt fonts
}{}
\usepackage[margin=1in]{geometry}
\usepackage{hyperref}
\hypersetup{unicode=true,
            pdftitle={spreadr: A R package to simulate spreading activation in a network},
            pdfauthor={Cynthia S. Q. Siew},
            pdfborder={0 0 0},
            breaklinks=true}
\urlstyle{same}  % don't use monospace font for urls
\usepackage{color}
\usepackage{fancyvrb}
\newcommand{\VerbBar}{|}
\newcommand{\VERB}{\Verb[commandchars=\\\{\}]}
\DefineVerbatimEnvironment{Highlighting}{Verbatim}{commandchars=\\\{\}}
% Add ',fontsize=\small' for more characters per line
\usepackage{framed}
\definecolor{shadecolor}{RGB}{248,248,248}
\newenvironment{Shaded}{\begin{snugshade}}{\end{snugshade}}
\newcommand{\KeywordTok}[1]{\textcolor[rgb]{0.13,0.29,0.53}{\textbf{#1}}}
\newcommand{\DataTypeTok}[1]{\textcolor[rgb]{0.13,0.29,0.53}{#1}}
\newcommand{\DecValTok}[1]{\textcolor[rgb]{0.00,0.00,0.81}{#1}}
\newcommand{\BaseNTok}[1]{\textcolor[rgb]{0.00,0.00,0.81}{#1}}
\newcommand{\FloatTok}[1]{\textcolor[rgb]{0.00,0.00,0.81}{#1}}
\newcommand{\ConstantTok}[1]{\textcolor[rgb]{0.00,0.00,0.00}{#1}}
\newcommand{\CharTok}[1]{\textcolor[rgb]{0.31,0.60,0.02}{#1}}
\newcommand{\SpecialCharTok}[1]{\textcolor[rgb]{0.00,0.00,0.00}{#1}}
\newcommand{\StringTok}[1]{\textcolor[rgb]{0.31,0.60,0.02}{#1}}
\newcommand{\VerbatimStringTok}[1]{\textcolor[rgb]{0.31,0.60,0.02}{#1}}
\newcommand{\SpecialStringTok}[1]{\textcolor[rgb]{0.31,0.60,0.02}{#1}}
\newcommand{\ImportTok}[1]{#1}
\newcommand{\CommentTok}[1]{\textcolor[rgb]{0.56,0.35,0.01}{\textit{#1}}}
\newcommand{\DocumentationTok}[1]{\textcolor[rgb]{0.56,0.35,0.01}{\textbf{\textit{#1}}}}
\newcommand{\AnnotationTok}[1]{\textcolor[rgb]{0.56,0.35,0.01}{\textbf{\textit{#1}}}}
\newcommand{\CommentVarTok}[1]{\textcolor[rgb]{0.56,0.35,0.01}{\textbf{\textit{#1}}}}
\newcommand{\OtherTok}[1]{\textcolor[rgb]{0.56,0.35,0.01}{#1}}
\newcommand{\FunctionTok}[1]{\textcolor[rgb]{0.00,0.00,0.00}{#1}}
\newcommand{\VariableTok}[1]{\textcolor[rgb]{0.00,0.00,0.00}{#1}}
\newcommand{\ControlFlowTok}[1]{\textcolor[rgb]{0.13,0.29,0.53}{\textbf{#1}}}
\newcommand{\OperatorTok}[1]{\textcolor[rgb]{0.81,0.36,0.00}{\textbf{#1}}}
\newcommand{\BuiltInTok}[1]{#1}
\newcommand{\ExtensionTok}[1]{#1}
\newcommand{\PreprocessorTok}[1]{\textcolor[rgb]{0.56,0.35,0.01}{\textit{#1}}}
\newcommand{\AttributeTok}[1]{\textcolor[rgb]{0.77,0.63,0.00}{#1}}
\newcommand{\RegionMarkerTok}[1]{#1}
\newcommand{\InformationTok}[1]{\textcolor[rgb]{0.56,0.35,0.01}{\textbf{\textit{#1}}}}
\newcommand{\WarningTok}[1]{\textcolor[rgb]{0.56,0.35,0.01}{\textbf{\textit{#1}}}}
\newcommand{\AlertTok}[1]{\textcolor[rgb]{0.94,0.16,0.16}{#1}}
\newcommand{\ErrorTok}[1]{\textcolor[rgb]{0.64,0.00,0.00}{\textbf{#1}}}
\newcommand{\NormalTok}[1]{#1}
\usepackage{graphicx,grffile}
\makeatletter
\def\maxwidth{\ifdim\Gin@nat@width>\linewidth\linewidth\else\Gin@nat@width\fi}
\def\maxheight{\ifdim\Gin@nat@height>\textheight\textheight\else\Gin@nat@height\fi}
\makeatother
% Scale images if necessary, so that they will not overflow the page
% margins by default, and it is still possible to overwrite the defaults
% using explicit options in \includegraphics[width, height, ...]{}
\setkeys{Gin}{width=\maxwidth,height=\maxheight,keepaspectratio}
\IfFileExists{parskip.sty}{%
\usepackage{parskip}
}{% else
\setlength{\parindent}{0pt}
\setlength{\parskip}{6pt plus 2pt minus 1pt}
}
\setlength{\emergencystretch}{3em}  % prevent overfull lines
\providecommand{\tightlist}{%
  \setlength{\itemsep}{0pt}\setlength{\parskip}{0pt}}
\setcounter{secnumdepth}{0}
% Redefines (sub)paragraphs to behave more like sections
\ifx\paragraph\undefined\else
\let\oldparagraph\paragraph
\renewcommand{\paragraph}[1]{\oldparagraph{#1}\mbox{}}
\fi
\ifx\subparagraph\undefined\else
\let\oldsubparagraph\subparagraph
\renewcommand{\subparagraph}[1]{\oldsubparagraph{#1}\mbox{}}
\fi

%%% Use protect on footnotes to avoid problems with footnotes in titles
\let\rmarkdownfootnote\footnote%
\def\footnote{\protect\rmarkdownfootnote}

%%% Change title format to be more compact
\usepackage{titling}

% Create subtitle command for use in maketitle
\newcommand{\subtitle}[1]{
  \posttitle{
    \begin{center}\large#1\end{center}
    }
}

\setlength{\droptitle}{-2em}

  \title{spreadr: A R package to simulate spreading activation in a network}
    \pretitle{\vspace{\droptitle}\centering\huge}
  \posttitle{\par}
    \author{Cynthia S. Q. Siew}
    \preauthor{\centering\large\emph}
  \postauthor{\par}
      \predate{\centering\large\emph}
  \postdate{\par}
    \date{2018-10-28}


\begin{document}
\maketitle

\subsection{Introduction}\label{introduction}

The notion of spreading activation is a prevalent metaphor in the
cognitive sciences; however, the tools to implement spreading activation
in a computational simulation are not as readily available. This paper
introduces the \texttt{spreadr} R package (pronunced `SPREAD-er'), which
can implement spreading activation within a specified network structure.
The algorithmic method implemented in \texttt{spreadr} subroutines
followed the approach described in Vitevitch, Ercal, and Adagarla
(2011), who viewed activation as a fixed cognitive resource that could
``spread'' among connected nodes in a network. See Vitevitch et al.
(2011) for more details on the implementation of the spreading
activation process.

\subsection{0: Set up}\label{set-up}

\begin{Shaded}
\begin{Highlighting}[]
\KeywordTok{options}\NormalTok{(}\DataTypeTok{stringsAsFactors =} \OtherTok{FALSE}\NormalTok{) }\CommentTok{# to prevent strings from being converted to a factor class}
\NormalTok{extrafont}\OperatorTok{::}\KeywordTok{loadfonts}\NormalTok{(}\DataTypeTok{quiet=}\OtherTok{TRUE}\NormalTok{)}

\CommentTok{# install.packages('devtools')}
\CommentTok{# library(devtools)}
\CommentTok{# install_github('csqsiew/spreadr') # download spreadr from my github page}
\KeywordTok{library}\NormalTok{(spreadr)}
\end{Highlighting}
\end{Shaded}

\begin{verbatim}
## Loading required package: Rcpp
\end{verbatim}

\subsection{1: Generate a network object for spreading activation to
occur
in}\label{generate-a-network-object-for-spreading-activation-to-occur-in}

First, the network on which the spreading of activation occurs on must
be specified. In this example, we use the \texttt{sample\_gnp} function
from the \texttt{igraph} R package to generate a network with 20 nodes
and undirected links are randomly placed between the nodes with a
probability of 0.2. In this step it is important that the
\texttt{network} argument in the \texttt{spreadr} function is either (i)
recognized by \texttt{igraph} as a network object and has a
\texttt{name} attribute (to specify meaningful node labels), or (ii) an
adjacency matrix (node names will simply be the column numbers).

Note that in this example \texttt{spreadr} implements spreading
activation process in an unweighted, undirected network, but it is
possible to implement the process in a weighted network as well whereby
more activation is passed between nodes that have ``stronger'' edges, or
in a directed (asymmetric) network where activation can pass from node i
to node j but not necessarily from node j to node i (additional examples
below).

\begin{Shaded}
\begin{Highlighting}[]
\KeywordTok{library}\NormalTok{(igraph)}
\end{Highlighting}
\end{Shaded}

\begin{verbatim}
## 
## Attaching package: 'igraph'
\end{verbatim}

\begin{verbatim}
## The following objects are masked from 'package:stats':
## 
##     decompose, spectrum
\end{verbatim}

\begin{verbatim}
## The following object is masked from 'package:base':
## 
##     union
\end{verbatim}

\begin{Shaded}
\begin{Highlighting}[]
\KeywordTok{set.seed}\NormalTok{(}\DecValTok{1}\NormalTok{)}

\NormalTok{g <-}\StringTok{ }\KeywordTok{sample_gnp}\NormalTok{(}\DecValTok{20}\NormalTok{, }\FloatTok{0.2}\NormalTok{, }\DataTypeTok{directed =}\NormalTok{ F, }\DataTypeTok{loops =}\NormalTok{ F) }\CommentTok{# make a random network}
\KeywordTok{V}\NormalTok{(g)}\OperatorTok{$}\NormalTok{name <-}\StringTok{ }\KeywordTok{paste0}\NormalTok{(}\StringTok{'N'}\NormalTok{, }\KeywordTok{as.character}\NormalTok{(}\DecValTok{1}\OperatorTok{:}\KeywordTok{gorder}\NormalTok{(g))) }\CommentTok{# give some meaningful labels for the 'name' attribute}

\KeywordTok{V}\NormalTok{(g)}\OperatorTok{$}\NormalTok{color <-}\StringTok{ }\KeywordTok{c}\NormalTok{(}\StringTok{'blue'}\NormalTok{, }\KeywordTok{rep}\NormalTok{(}\StringTok{'white'}\NormalTok{, }\DecValTok{19}\NormalTok{))}
\NormalTok{l <-}\StringTok{ }\KeywordTok{layout_with_fr}\NormalTok{(g) }\CommentTok{# to save the layout }
\KeywordTok{plot}\NormalTok{(g, }\DataTypeTok{l =}\NormalTok{ l, }\DataTypeTok{vertex.size =} \DecValTok{10}\NormalTok{, }\DataTypeTok{vertex.label.dist =} \DecValTok{2}\NormalTok{, }\DataTypeTok{vertex.label.cex =} \FloatTok{0.8}\NormalTok{)}
\end{Highlighting}
\end{Shaded}

\includegraphics{spreadr_vignette_files/figure-latex/unnamed-chunk-2-1.pdf}

\begin{Shaded}
\begin{Highlighting}[]
\CommentTok{# the blue node will be assigned activation at t = 0}
\end{Highlighting}
\end{Shaded}

\subsection{2: Create a dataframe with initial activation
values}\label{create-a-dataframe-with-initial-activation-values}

The user must then specify the initial activation level(s) of node(s) in
the network in a dataframe object with two columns labeled node and
activation. Below the node labeled ``N1'' was assigned 20 units of
activation. The user can choose to provide different activation values,
or initialize more nodes with various activation values.

\begin{Shaded}
\begin{Highlighting}[]
\NormalTok{initial_df <-}\StringTok{ }\KeywordTok{data.frame}\NormalTok{(}\DataTypeTok{node =} \StringTok{'N1'}\NormalTok{, }\DataTypeTok{activation =} \DecValTok{20}\NormalTok{, }\DataTypeTok{stringsAsFactors =}\NormalTok{ F)}
\NormalTok{initial_df}
\end{Highlighting}
\end{Shaded}

\begin{verbatim}
##   node activation
## 1   N1         20
\end{verbatim}

\subsection{3: Run the simulation}\label{run-the-simulation}

We are finally ready to run the simulation. In this step, the user must
specify the following arguments and parameters in the \texttt{spreadr}
function:

\begin{enumerate}
\def\labelenumi{(\roman{enumi})}
\tightlist
\item
  \emph{start\_run}: the dataframe (\texttt{initial\_df}) specified in
  the previous step that contains the activation values assigned to
  nodes at t = 0;
\item
  \emph{decay}, d: the proportion of activation lost at each time step
  (range from 0 to 1);
\item
  \emph{retention}, r: the proportion of activation retained in the
  originator node (range from 0 to 1);
\item
  \emph{suppress}, d: nodes with activation values lower than this value
  will have their activations forced to 0. Typically this will be a very
  small value (e.g., \textless{} 0.001);
\item
  \emph{network}: the network (must be an igraph object or a non-zero
  matrix) on which the spreading of activation occurs on, and
\item
  \emph{time}, t: the number of times to run the spreading activation
  process for.\\
\item
  \emph{create\_name}: creates numbers/names for nodes if needed,
  default is TRUE.
\end{enumerate}

\begin{Shaded}
\begin{Highlighting}[]
\NormalTok{result <-}\StringTok{ }\NormalTok{spreadr}\OperatorTok{::}\KeywordTok{spreadr}\NormalTok{(}\DataTypeTok{start_run =}\NormalTok{ initial_df, }\DataTypeTok{decay =} \DecValTok{0}\NormalTok{,}
                              \DataTypeTok{retention =} \FloatTok{0.5}\NormalTok{, }\DataTypeTok{suppress =} \DecValTok{0}\NormalTok{,}
                              \DataTypeTok{network =}\NormalTok{ g, }\DataTypeTok{time =} \DecValTok{10}\NormalTok{) }
\end{Highlighting}
\end{Shaded}

\subsection{4: Results}\label{results}

The output is a dataframe with 3 columns labeled \emph{node},
\emph{activation}, and \emph{time}, and contains the activation value of
each node at each time step. The output can be easily saved as a .csv
file for further analysis later. A plot showing the activation levels of
each node in the network at each time step is shown below.

\begin{Shaded}
\begin{Highlighting}[]
\KeywordTok{head}\NormalTok{(result, }\DecValTok{10}\NormalTok{) }\CommentTok{# view the results}
\end{Highlighting}
\end{Shaded}

\begin{verbatim}
##    node activation time
## 1    N1         10    1
## 2    N2          0    1
## 3    N3          0    1
## 4    N4          0    1
## 5    N5          0    1
## 6    N6          0    1
## 7    N7          0    1
## 8    N8          0    1
## 9    N9          0    1
## 10  N10          0    1
\end{verbatim}

\begin{Shaded}
\begin{Highlighting}[]
\KeywordTok{tail}\NormalTok{(result, }\DecValTok{10}\NormalTok{)}
\end{Highlighting}
\end{Shaded}

\begin{verbatim}
##     node activation time
## 191  N11  1.3281320   10
## 192  N12  0.1420159   10
## 193  N13  0.2170434   10
## 194  N14  1.4254933   10
## 195  N15  1.1792312   10
## 196  N16  0.9448024   10
## 197  N17  0.6828021   10
## 198  N18  1.5931093   10
## 199  N19  1.2969978   10
## 200  N20  1.6084249   10
\end{verbatim}

\begin{Shaded}
\begin{Highlighting}[]
\CommentTok{# write.csv(result, file = 'result.csv') # save the results }

\KeywordTok{library}\NormalTok{(ggplot2) }
\NormalTok{a1 <-}\StringTok{ }\KeywordTok{data.frame}\NormalTok{(}\DataTypeTok{node =} \StringTok{'N1'}\NormalTok{, }\DataTypeTok{activation =} \DecValTok{20}\NormalTok{, }\DataTypeTok{time =} \DecValTok{0}\NormalTok{) }\CommentTok{# add back initial activation at t = 0}
\NormalTok{result_t0 <-}\StringTok{ }\KeywordTok{rbind}\NormalTok{(result,a1)}
\KeywordTok{ggplot}\NormalTok{(}\DataTypeTok{data =}\NormalTok{ result_t0, }\KeywordTok{aes}\NormalTok{(}\DataTypeTok{x =}\NormalTok{ time, }\DataTypeTok{y =}\NormalTok{ activation, }\DataTypeTok{color =}\NormalTok{ node, }\DataTypeTok{group =}\NormalTok{ node)) }\OperatorTok{+}
\StringTok{  }\KeywordTok{geom_point}\NormalTok{() }\OperatorTok{+}\StringTok{ }\KeywordTok{geom_line}\NormalTok{() }\OperatorTok{+}\StringTok{ }\KeywordTok{ggtitle}\NormalTok{(}\StringTok{'unweighted, undirected network'}\NormalTok{) }\CommentTok{# visualize the results }
\end{Highlighting}
\end{Shaded}

\includegraphics{spreadr_vignette_files/figure-latex/unnamed-chunk-5-1.pdf}

\subsection{5: Additional examples}\label{additional-examples}

This section shows how spreadr can be implemented in a weighted network
(i.e., the edges in the network can have different weights) and a
directed network (i.e., the edges in the network are not symmetric).

\begin{enumerate}
\def\labelenumi{(\roman{enumi})}
\tightlist
\item
  Weighted, undirected network
\end{enumerate}

\begin{Shaded}
\begin{Highlighting}[]
\CommentTok{# weighted, undirected network example }
\NormalTok{g_w <-}\StringTok{ }\NormalTok{g}
\KeywordTok{set.seed}\NormalTok{(}\DecValTok{2}\NormalTok{)}
\KeywordTok{E}\NormalTok{(g_w)}\OperatorTok{$}\NormalTok{weight <-}\StringTok{ }\KeywordTok{runif}\NormalTok{(}\KeywordTok{gsize}\NormalTok{(g_w)) }\CommentTok{# make the edges in the network have different weights ranging from 0 to 1 (excluding extreme values)}

\NormalTok{result_w <-}\StringTok{ }\NormalTok{spreadr}\OperatorTok{::}\KeywordTok{spreadr}\NormalTok{(}\DataTypeTok{start_run =}\NormalTok{ initial_df, }\DataTypeTok{decay =} \DecValTok{0}\NormalTok{,}
                              \DataTypeTok{retention =} \FloatTok{0.5}\NormalTok{, }\DataTypeTok{suppress =} \DecValTok{0}\NormalTok{,}
                              \DataTypeTok{network =}\NormalTok{ g_w, }\DataTypeTok{time =} \DecValTok{10}\NormalTok{) }

\NormalTok{result_w_t0 <-}\StringTok{ }\KeywordTok{rbind}\NormalTok{(result_w, a1) }\CommentTok{# add back initial activation at t = 0}
\KeywordTok{ggplot}\NormalTok{(}\DataTypeTok{data =}\NormalTok{ result_w_t0, }\KeywordTok{aes}\NormalTok{(}\DataTypeTok{x =}\NormalTok{ time, }\DataTypeTok{y =}\NormalTok{ activation, }\DataTypeTok{color =}\NormalTok{ node, }\DataTypeTok{group =}\NormalTok{ node)) }\OperatorTok{+}
\StringTok{  }\KeywordTok{geom_point}\NormalTok{() }\OperatorTok{+}\StringTok{ }\KeywordTok{geom_line}\NormalTok{() }\OperatorTok{+}\StringTok{ }\KeywordTok{ggtitle}\NormalTok{(}\StringTok{'weighted, undirected network'}\NormalTok{) }\CommentTok{# visualize the results }
\end{Highlighting}
\end{Shaded}

\includegraphics{spreadr_vignette_files/figure-latex/unnamed-chunk-6-1.pdf}

As you can see from the plot, the result is slightly different from the
original example with unweighted edges. More activation is passed to the
node that is more ``strongly'' connected to the originator node.

\begin{enumerate}
\def\labelenumi{(\roman{enumi})}
\setcounter{enumi}{1}
\tightlist
\item
  Unweighted, directed network
\end{enumerate}

For this example, we will create the network via an adjacency matrix.

\begin{Shaded}
\begin{Highlighting}[]
\CommentTok{# unweighted, directed network example}
\KeywordTok{set.seed}\NormalTok{(}\DecValTok{3}\NormalTok{)}
\NormalTok{g_d_mat <-}\StringTok{ }\KeywordTok{matrix}\NormalTok{(}\KeywordTok{sample}\NormalTok{(}\KeywordTok{c}\NormalTok{(}\DecValTok{0}\NormalTok{,}\DecValTok{1}\NormalTok{), }\DecValTok{100}\NormalTok{, }\DataTypeTok{replace =}\NormalTok{ T), }\DecValTok{10}\NormalTok{, }\DecValTok{10}\NormalTok{) }\CommentTok{# make a matrix and randomly fill some cells with 1s }
\KeywordTok{diag}\NormalTok{(g_d_mat) <-}\StringTok{ }\DecValTok{0} \CommentTok{# remove self-loops }
\NormalTok{g_d_mat}
\end{Highlighting}
\end{Shaded}

\begin{verbatim}
##       [,1] [,2] [,3] [,4] [,5] [,6] [,7] [,8] [,9] [,10]
##  [1,]    0    1    0    0    0    0    1    1    1     0
##  [2,]    1    0    0    0    1    0    0    1    1     0
##  [3,]    0    1    0    0    0    1    1    1    0     0
##  [4,]    0    1    0    0    1    1    0    1    0     1
##  [5,]    1    1    0    0    0    1    1    1    1     1
##  [6,]    1    1    1    0    0    0    0    1    1     0
##  [7,]    0    0    1    1    0    0    0    0    0     0
##  [8,]    0    1    1    0    0    0    1    0    0     0
##  [9,]    1    1    1    1    0    0    0    1    0     0
## [10,]    1    0    1    0    1    0    0    1    1     0
\end{verbatim}

\begin{Shaded}
\begin{Highlighting}[]
\CommentTok{# spreadr will work on an adjacency matrix too }
\NormalTok{result_d <-}\StringTok{ }\NormalTok{spreadr}\OperatorTok{::}\KeywordTok{spreadr}\NormalTok{(}\DataTypeTok{start_run =} \KeywordTok{data.frame}\NormalTok{(}\DataTypeTok{node =} \DecValTok{1}\NormalTok{, }\DataTypeTok{activation =} \DecValTok{20}\NormalTok{, }\DataTypeTok{stringsAsFactors =}\NormalTok{ F), }
                             \DataTypeTok{decay =} \DecValTok{0}\NormalTok{,}
                              \DataTypeTok{retention =} \FloatTok{0.5}\NormalTok{, }\DataTypeTok{suppress =} \DecValTok{0}\NormalTok{,}
                              \DataTypeTok{network =}\NormalTok{ g_d_mat, }\DataTypeTok{time =} \DecValTok{10}\NormalTok{) }

\NormalTok{result_d_t0 <-}\StringTok{ }\KeywordTok{rbind}\NormalTok{(result_d, }\KeywordTok{data.frame}\NormalTok{(}\DataTypeTok{node =} \DecValTok{1}\NormalTok{, }\DataTypeTok{activation =} \DecValTok{20}\NormalTok{, }\DataTypeTok{time =} \DecValTok{0}\NormalTok{)) }\CommentTok{# add back initial activation at t = 0}
\KeywordTok{ggplot}\NormalTok{(}\DataTypeTok{data =}\NormalTok{ result_d_t0, }\KeywordTok{aes}\NormalTok{(}\DataTypeTok{x =}\NormalTok{ time, }\DataTypeTok{y =}\NormalTok{ activation, }\DataTypeTok{color =}\NormalTok{ node, }\DataTypeTok{group =}\NormalTok{ node)) }\OperatorTok{+}
\StringTok{  }\KeywordTok{geom_point}\NormalTok{() }\OperatorTok{+}\StringTok{ }\KeywordTok{geom_line}\NormalTok{() }\OperatorTok{+}\StringTok{ }\KeywordTok{ggtitle}\NormalTok{(}\StringTok{'unweighted, directed network'}\NormalTok{) }
\end{Highlighting}
\end{Shaded}

\includegraphics{spreadr_vignette_files/figure-latex/unnamed-chunk-7-1.pdf}

It is important to keep in mind that the directionality of the edges
goes column-wise and not row-wise. In other words, node 1
--\textgreater{} node 2 is represented by filling in the cell in COLUMN
1, ROW 2 of the adjacency matrix. See below and compare it to the plot
above.

\begin{Shaded}
\begin{Highlighting}[]
\CommentTok{# first column }
\KeywordTok{data.frame}\NormalTok{(}\DataTypeTok{node =} \DecValTok{1}\OperatorTok{:}\DecValTok{10}\NormalTok{, }\DataTypeTok{edge =}\NormalTok{ g_d_mat[,}\DecValTok{1}\NormalTok{])}
\end{Highlighting}
\end{Shaded}

\begin{verbatim}
##    node edge
## 1     1    0
## 2     2    1
## 3     3    0
## 4     4    0
## 5     5    1
## 6     6    1
## 7     7    0
## 8     8    0
## 9     9    1
## 10   10    1
\end{verbatim}

\begin{Shaded}
\begin{Highlighting}[]
\CommentTok{# hence activation is going from node 1 to nodes 3,4,10 - as in the plot }

\CommentTok{# first row - wrong direction!}
\KeywordTok{data.frame}\NormalTok{(}\DataTypeTok{node =} \DecValTok{1}\OperatorTok{:}\DecValTok{10}\NormalTok{, }\DataTypeTok{edge =}\NormalTok{ g_d_mat[}\DecValTok{1}\NormalTok{,])}
\end{Highlighting}
\end{Shaded}

\begin{verbatim}
##    node edge
## 1     1    0
## 2     2    1
## 3     3    0
## 4     4    0
## 5     5    0
## 6     6    0
## 7     7    1
## 8     8    1
## 9     9    1
## 10   10    0
\end{verbatim}

\begin{Shaded}
\begin{Highlighting}[]
\CommentTok{# activation does NOT go from node 1 to nodes 3,4,5,7,8,10}
\end{Highlighting}
\end{Shaded}

\subsubsection{References}\label{references}

Siew, C. S. Q. (under review). spreadr: A R package to simulate
spreading activation in a network.\\
Vitevitch, M. S., Ercal, G., \& Adagarla, B. (2011). Simulating
retrieval from a highly clustered network: Implications for spoken word
recognition. \emph{Frontiers in Psychology, 2}, 369.


\end{document}
